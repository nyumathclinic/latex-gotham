%
% \iffalse
%<*driver>
\ProvidesFile{beamerfontthemeNYU22.dtx}
%</driver>
%<pkg>\ProvidesPackage{beamerfontthemeNYU22}
%<*pkg>
  [2025/02/07 v0.1 A beamer font theme for the 2022 NYU brand]
%</pkg>
%<*driver>
\documentclass[11pt]{ltxdoc}
\usepackage[letterpaper]{geometry}
\usepackage{changepage}^^A      For the `adjustwidth' environment
\usepackage{hologo}
    
\usepackage[backend=biber]{biblatex}
\addbibresource{main.bib}
\addbibresource{beamerfontthemeNYU22.bib}


\usepackage[
    Gotham, Frank Ruhl Libre,
    tone=traditional-subtle]{nyu22fonts}
\linespread{1.041667}
\setmonofont{Inconsolata}
\usepackage{parskip}

\usepackage[title/color=NyuViolet]{xcolor-nyu22}

\setcounter{secnumdepth}{2}


\usepackage{dtxdescribe}
\colorlet{\watchoutcolor}{Magenta}

\DisableCrossrefs
\CodelineIndex
\RecordChanges
\AtBeginDocument{
  \hypersetup{
    allcolors=NyuViolet,
  }
}
\begin{document}
  \DocInput{beamerfontthemeNYU22.dtx}
\end{document}
%</driver>
% \fi
%
% \GetFileInfo{beamerfontthemeNYU22.dtx}
% \title{A Beamer Font Theme for the 2022 NYU Brand}
% \author{Matthew Leingang\thanks{leingang@nyu.edu}}
% \date{\fileversion, Released \filedate}
% \maketitle
%
% \begin{abstract}
% This package provides a beamer font theme that aligns with the 2022 edition of
% the NYU brand identity, including fonts~\cite{nyu-fonts} and colors~\cite{nyu-colors}.
% \end{abstract}
%
% \changes{v0.1}{2025/02/07}{First working release}%
%
% \section{Introduction}
% \begin{refsection}
%
% This beamer theme is designed to align with the 2022 edition of the NYU brand
% identity, including fonts~\cite{nyu-fonts} and colors~\cite{nyu-colors}.
%
% \printbibliography[heading=subbibliography]
% \end{refsection}
%
%
% \section{Implementation}
% \begin{refsection}
%
%    \begin{macrocode}
%<*pkg>
%    \end{macrocode}
%
% Need \pkg{fontspec} to set the fonts.
%    \begin{macrocode}
\RequirePackage{fontspec}
%    \end{macrocode}
% 
% A monospace font that goes well with Gotham is Inconsolata.
%    \begin{macrocode}
\setmonofont{Inconsolata}
%    \end{macrocode}
% 
% We need to load \pkg{mathtools} before \pkg{unicode-math} to avoid a symbol clash.
% See \url{https://tex.stackexchange.com/a/736498/1402}.
%    \begin{macrocode}
\RequirePackage{mathtools}
\RequirePackage[mathrm=sym]{unicode-math}
%    \end{macrocode}
% 
% This is the default math OpenType font.
% At some point we may want to make this configurable, or choose something
% more compatible with Gotham.
%    \begin{macrocode}
\setmathfont{latinmodern-math.otf}
%    \end{macrocode}
% 
% Beamer takes pains to make math on slides look readable, which means the
% default math fonts are replaced with sans-serif versions. 
% A ``professional'' font package like \pkg{unicode-math} skip these
% replacements. So we have to load them back in.
%    \begin{macrocode}
\setmathfont[range=it/{latin,Latin}]{Gotham Book Italic}
\setmathfont[range=bfup/{latin,Latin}]{Gotham Medium}
\setmathfont[range=up/{num}]{Gotham Book}
%    \end{macrocode}
%
%
% That's all, folks!
%    \begin{macrocode}
%</pkg>
%    \end{macrocode}
%
% \printbibliography[
%   heading=subbibliography,
%   title={Implementation References}]
% \end{refsection}
%
% \Finale
